\documentclass[aspectratio=169,xcolor={dvipsnames,svgnames,table}]{beamer}

\usepackage{centernot}
\usepackage[utf8]{inputenc}
\beamertemplatenavigationsymbolsempty
\title{Anwendung des Folgenkriteriums}
\date{}

\newcommand*{\N}{\mathbb N}
\newcommand*{\Z}{\mathbb Z}
\newcommand*{\Q}{\mathbb Q}
\newcommand*{\R}{\mathbb R}
\newcommand*{\C}{\mathbb C}

\begin{document}
  \begin{frame}
    \titlepage
  \end{frame}

  \begin{frame}
    \frametitle{Aufgabe}

    Berechne $\lim_{n\to\infty} \left(1+\frac 1n \right)^2$
  \end{frame}

  \begin{frame}
    \frametitle{Lösung}

    \begin{align*}
      \uncover<1->{& \lim_{n\to\infty} \left(1+\frac 1n\right)^2} \\[0.3em]
      \uncover<2->{&{\color{Gray}\left\downarrow\ \text{Quadratfunktion ist stetig}\right.}} \\[0.3em]
      \uncover<3->{=\ & \left(\lim_{n\to\infty} \left(1+\frac 1n\right)\right)^2} \\[0.3em]
      \uncover<4->{&{\color{Gray}\left\downarrow\ \lim_{n\to\infty} \left( a_n+ b_n \right) = \lim_{n\to\infty} a_n+ \lim_{n\to\infty} b_n\right.}} \\[0.3em]
      \uncover<5->{=\ &\left( \lim_{n\to\infty} 1 + \lim_{n \to \infty} \frac 1n \right)^2} \\[0.3em]
      \uncover<6->{=\ & (1+0)^2 = 1}
    \end{align*}
  \end{frame}

  \begin{frame}

  \end{frame}
\end{document}
